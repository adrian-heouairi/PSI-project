Il s'agit projet réalisé en Go 1.21. Le programme présente une interface en ligne de commande avec TAB completion pour les noms de commande et leurs arguments (voir liste des commandes ci-dessous). On peut également lancer une commande avec par exemple \mintinline{bash}{go run . wget jch.irif.fr/images/horse.jpg} (le programme se termine après l'avoir exécutée).

\begin{minted}{bash}
> help
PATH is PEER_NAME[PATH2] with PATH2 = /videos for example
    hello PEER: sends at least two Hellos to PEER
    exit: exits the program
    help: shows help message
    lspeers: shows the connected peers if --addr specified shows also addresses
    findrem PEER: shows the files shared by PEER
    curl PATH: downloads and shows the file at PATH
    wget PATH: downloads recursively the directory or file at PATH
\end{minted}
Le projet est découpé en fichiers \mintinline{bash}{.go} : \mintinline{bash}{constants.go download.go main.go merkle_tree.go}\\
\mintinline{bash}{messages.go rest.go udp.g ui.go utils.go}.

